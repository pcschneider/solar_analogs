\documentclass[oldversion]{aa}  

\usepackage{fp, fp-exp}
\usepackage{xcolor}
%
\usepackage{graphicx}
%%%%%%%%%%%%%%%%%%%%%%%%%%%%%%%%%%%%%%%%
\usepackage{txfonts}
% \usepackage{tablefootnote}
% 
% \usepackage{footnote}
% \makesavenoteenv{tabular}
% \makesavenoteenv{table}
\usepackage{natbib}
\bibpunct{(}{)}{;}{a}{}{,} % to follow the A&A style
%%%%%%%%%%%%%%%%%%%%%%%%%%%%%%%%%%%%%%%%
%\usepackage[options]{hyperref}
% To add links in your PDF file, use the package "hyperref"
% with options according to your LaTeX or PDFLaTeX drivers.
%
\begin{document} 

\title{X-ray properties of Solar Analogs\thanks{Based on observations obtained with XMM-Newton, an ESA science mission with instruments and contributions directly funded by ESA Member States and NASA, and based on observations obtained by the Chandra X-ray observatory.}}

%     \subtitle{X-ray vs optical and UV accretion tracers}

   \author{P. C. Schneider\inst{1}
          \and
          H. M. G\"unther\inst{2}
          \and
          J. Robrade\inst{1}
          \and
          S. Czesla\inst{1}
          \and
          M. Salz\inst{1}
          \and
          J. H. M. M. Schmitt\inst{1}
          }

   \institute{ Hamburger Sternwarte, Gojenbergsweg 112,
              21029 Hamburg, Germany
               \email{astro@pcschneider.eu}
     \and  
              Massachusetts Institute of Technology,
            Kavli Institute for Astrophysics \& Space Research,
            77 Massachusetts Avenue, Cambridge, MA 02139, USA
         }
     
               
   \date{received; accepted}

% \abstract{}{}{}{}{} 
% 5 {} token are mandatory
 
  \abstract
   {
TBD
     }

   \keywords{Stars: low-mass, X-rays: stars}

   \maketitle
   
\section{Introduction}
Solar Analogs are


\begin{table*}
  \centering
  \begin{tabular}{rccccccccl}
    \hline
    \hline
    Target & Mass      & \multicolumn{3}{c}{Age} & $\log R'_{\mathrm{HK}}$ & Vmag & Gmag &  Distance & Refs\\
           & $M_\odot$ & min & nominal & max     &                       &      &      & (pc) & \\ 
    \hline       
HD 135101 & 0.99$\pm$0.01 & 9.30 & 9.60 & 9.90 & -5.04 & 6.68 & 6.49 & 32.36 & 1, 2 \\ 
HD 157347 & 0.97$\pm$0.01 & 5.90 & 6.70 & 7.20 & -5.01 & 6.29 & 6.10 & 19.66 & 1, 2 \\ 
HD 45289 & 0.98$\pm$0.00 & 9.30 & 9.50 & 9.80 & -5.01 & 6.65 & 6.49 & 27.87 & 1, 2 \\ 
HD 210918 & 0.96$\pm$0.00 & 8.80 & 9.20 & 9.60 & -5.00 & 6.23 & 6.05 &  nan & 1, 2 \\ 
HD 39881 & 0.96$\pm$0.00 & 9.40 & 9.70 & 10.00 & -5.00 & 6.60 & 6.44 & 27.51 & 1, 2 \\ 
HD 25874 & 0.99$\pm$0.00 & 6.90 & 7.30 & 7.70 & -5.00 & 6.73 & 6.56 & 25.97 & 1, 2 \\ 
HD 114174 & 0.99$\pm$0.00 & 5.70 & 6.40 & 7.00 & -4.99 & 6.82 & 6.61 & 26.38 & 1, 2 \\ 
HD 146233 & 1.04$\pm$0.01 & 2.40 & 3.00 & 3.30 & -4.98 & 5.51 & 5.30 & 14.13 & 1, 2 \\ 
HD 150248 & 0.96$\pm$0.00 & 6.80 & 7.60 & 8.00 & -4.96 & 7.03 & 6.85 & 27.79 & 1, 2 \\ 
HD 42618 & 0.97$\pm$0.01 & 4.60 & 5.40 & 6.00 & -4.95 & 6.87 & 6.68 & 24.35 & 1, 2 \\ 
HD 2071 & 0.97$\pm$0.01 & 3.90 & 4.70 & 5.70 & -4.93 & 7.27 & 7.11 & 28.35 & 1, 2 \\ 
HD 120690 & 0.95$\pm$0.01 & 6.20 & 7.50 & 8.30 & -4.80 & 6.44 & 6.24 & 18.56 & 1, 2 \\ 
HD 1461 &  nan$\pm$ nan &  nan &  nan &  nan &  nan &  nan & 6.29 & 23.47 & 1, 2 \\ 
HD 190771 &  nan$\pm$ nan &  nan &  nan &  nan &  nan &  nan & 5.99 & 19.02 & 1, 2 \\ 

    \hline
  \end{tabular}
  
   \tablebib{
(1)~\citet{Ramirez_2014}; (2) \citet{Gaia_2016}; \citet{Gaia_2018};
}
\end{table*}


\section{Observations and Data Reduction}
The X-ray observations of solar analogs were obtained over several years with XMM-Newton and Chandra. Table~\ref{tab:obs} provides an overview of the analyzed data.

\begin{table}
\caption{Log of X-ray observations\label{tab:obs}}

  \centering
  \begin{tabular}{rcccr}
    \hline
    \hline
    Target & Observatory & Obs ID & Date & Exp Time \\
           &              &       &      & (ks) \\ 
    \hline       
HD 1461 & XMM & 0804040401 & 2017.416 &  46.8 \\ 
HD 2071 & XMM & 0784240101 & 2016.402 &   9.9 \\ 
HD 25874 & XMM & 0784240501 & 2016.718 &  11.9 \\
HD 25874 & Chandra & 22357 & unobs. & 25\\
HD 39881 & XMM & 0784241301 & 2016.778 &  16.9 \\ 
HD 42618 & XMM & 0784240201 & 2016.743 &   9.8 \\ 
HD 45289 & XMM & 0784241101 & 2016.391 &  15.9 \\ 
HD 114174 & XMM & 0784240301 & 2016.546 &  19.1 \\
HD 114174 & Chandra & 22356 & 2020.265 & 24.8 \\
HD 120690 & XMM & 0784240701 & 2017.013 &  11.9 \\ 
HD 135101 & XMM & 0784241201 & 2016.543 &  21.9 \\ 
HD 150248 & XMM & 0405180501 & 2011.147 &  39.6 \\ 
HD 157347 & XMM & 0804040301 & 2017.652 &  35.8 \\ 
HD 157347 & Chandra & 22355 & 2020.071 & 14.9 \\
HD 190771 & XMM & 0804040201 & 2017.652 &  35.8 \\ 
HD 210918 & XMM & 0784240901 & 2016.358 &  15.9 \\ 
HD 210918 & Chandra & 22358 & 2019.986 & 20.6\\
    \hline
  \end{tabular}
\end{table}


\bibliographystyle{aa} 
\bibliography{analogs}
\end{document}
